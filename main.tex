%%%%%%%%%%%%%%%%%%%%%%%%%%%%%%%%%%%%%%%%%
% Developer CV
% LaTeX Template
% Version 1.0 (28/1/19)
%
% This template originates from:
% http://www.LaTeXTemplates.com
%
% Authors:
% Jan Vorisek (jan@vorisek.me)
% Based on a template by Jan Küster (info@jankuester.com)
% Modified for LaTeX Templates by Vel (vel@LaTeXTemplates.com)
%
% License:
% The MIT License (see included LICENSE file)
%
%%%%%%%%%%%%%%%%%%%%%%%%%%%%%%%%%%%%%%%%%

%----------------------------------------------------------------------------------------
%	PACKAGES AND OTHER DOCUMENT CONFIGURATIONS
%----------------------------------------------------------------------------------------

\documentclass[9pt]{developercv} % Default font size, values from 8-12pt are recommended

%----------------------------------------------------------------------------------------

\begin{document}

%----------------------------------------------------------------------------------------
%	TITLE AND CONTACT INFORMATION
%----------------------------------------------------------------------------------------

\begin{minipage}[t]{0.45\textwidth} % 45% of the page width for name
	\vspace{-\baselineskip} % Required for vertically aligning minipages
	
	% If your name is very short, use just one of the lines below
	% If your name is very long, reduce the font size or make the minipage wider and reduce the others proportionately
	\colorbox{black}{{\HUGE\textcolor{white}{\textbf{\MakeUppercase{Kar Rui}}}}} % First name
	
	\colorbox{black}{{\HUGE\textcolor{white}{\textbf{\MakeUppercase{Lau}}}}} % Last name
	
	\vspace{6pt}
	
	% {\huge Software Engineer} % Career or current job title
\end{minipage}
\begin{minipage}[t]{0.275\textwidth} % 27.5% of the page width for the first row of icons
	\vspace{-\baselineskip} % Required for vertically aligning minipages
	
	% The first parameter is the FontAwesome icon name, the second is the box size and the third is the text
	% Other icons can be found by referring to fontawesome.pdf (supplied with the template) and using the word after \fa in the command for the icon you want
	\icon{MapMarker}{12}{Singapore}\\
	\icon{Phone}{12}{+65 83221353}\\
	\icon{At}{12}{\href{mailto:hello@karrui.dev}{hello@karrui.dev}}\\	
\end{minipage}
\begin{minipage}[t]{0.275\textwidth} % 27.5% of the page width for the second row of icons
	\vspace{-\baselineskip} % Required for vertically aligning minipages
	
	% The first parameter is the FontAwesome icon name, the second is the box size and the third is the text
	% Other icons can be found by referring to fontawesome.pdf (supplied with the template) and using the word after \fa in the command for the icon you want
	\icon{Globe}{12}{\href{https://karrui.dev}{karrui.dev}}\\
	\icon{Github}{12}{\href{https://github.com/karrui}{github.com/karrui}}\\
	\icon{Linkedin}{12}{\href{https://linkedin.com/in/karrui/}{linkedin.com/in/karrui}}\\
\end{minipage}

\vspace{0.5cm}

%----------------------------------------------------------------------------------------
%	INTRODUCTION, SKILLS AND TECHNOLOGIES
%----------------------------------------------------------------------------------------

% \cvsect{Who Am I?}

% \begin{minipage}[t]{0.4\textwidth} % 40% of the page width for the introduction text
% 	\vspace{-\baselineskip} % Required for vertically aligning minipages
	
% 	\lorem \lorem \lorem \lorem \lorem\\ % Dummy text
% \end{minipage}
% \hfill % Whitespace between
% \begin{minipage}[t]{0.5\textwidth} % 50% of the page for the skills bar chart
% 	\vspace{-\baselineskip} % Required for vertically aligning minipages
% 	\begin{barchart}{5.5}
% 		\baritem{JavaScript}{60}
% 		\baritem{PHP}{100}
% 		\baritem{SASS/LESS}{70}
% 		\baritem{Bootstrap}{70}
% 		\baritem{Git}{40}
% 		\baritem{LaTeX}{60}
% 	\end{barchart}
% \end{minipage}

% \begin{center}
% 	\bubbles{5/Eclipse, 6/git, 4/Office, 3/Inkscape, 3/Blender}
% \end{center}

%----------------------------------------------------------------------------------------
%	EXPERIENCE
%----------------------------------------------------------------------------------------

\cvsect{Experience}

\begin{entrylist}
	\entry
		{Feb 2020 -- Now\\\footnotesize{fulltime}}
		{Senior Software Engineer}
		{\href{https://open.gov.sg}{Open Government Products}}
		{
      I mainly work on \underline{\href{https://form.gov.sg}{FormSG}}, an official Singapore Government's form builder application. I was also part of the team tasked to work on Singapore's COVID-19 response.
			\begin{itemize}[noitemsep, leftmargin=1.25em]
				\item Epic lead for a frontend framework migration from AngularJS to React for \underline{\href{https://github.com/opengovsg/formsg}{FormSG}}. 
        \item Epic lead for an API migration, which includes renaming, simplifying, and refactoring endpoints to a more RESTful and meaningful state.
        \item Spearheaded the creation of an internal design system, released as an \underline{\href{https://www.npmjs.com/package/@opengovsg/design-system-react}{npm package}} with close collaboration with the designers to achieve visual continuity and consistency across OGP products (eventually).
        \item Promoted the use of Storybook and Chromatic as a visual regression testing and usage-guidelines tool to other software engineers.
        \item Built the entire clinician-facing frontend application for Singapore's COVID-19 Appointment System to allow clinic staff to manage appointments and patients coming in for their vaccinations.
        \item Wrote lambda pipelines to automate parts of patient and case management relating to Singapore's COVID-19 Endemic response.
			\end{itemize}
		\texttt{Node.js}\slashsep\texttt{React.js}\slashsep\texttt{TypeScript}\slashsep\texttt{JavaScript}\slashsep\texttt{Design Systems}}
	\entry
		{May -- Aug 2019\\\footnotesize{internship}}
		{Software Engineer Intern, Google Pay}
		{Google Singapore}
		{Completed and had full ownership of a new feature in the (then) Android application, including but not limited to high level architecture, design decisions, implementation, and testing.
		\\ \texttt{Java}\slashsep\texttt{Android}}
	\entry
		{May -- Aug 2018\\\footnotesize{internship}}
		{Frontend Software Engineer Intern}
		{\href{https://intelllex.com}{INTELLLEX}}
		{
			Implement features and components related to the design overhaul of the main enterprise React app. \\
			In addition, I also refactored many tightly coupled components into pure functional components, resulting in improved readability and testability.
			% % \setlist{nolistsep}
    	% \begin{itemize}[noitemsep, leftmargin=1.25em]
			% 	\item Implement features and components related to the design overhaul of the main enterprise app. 
			% 	\item Refactor many tightly coupled components into pure functional components, resulting in improved readability and testability.
			% \end{itemize}
			\\ \texttt{JavaScript}\slashsep\texttt{React}
		}
\end{entrylist}

%----------------------------------------------------------------------------------------
%	EDUCATION
%----------------------------------------------------------------------------------------

\cvsect{Education}

\begin{entrylist}
	\entry
		{2016 -- 2019}
		{B.Comp (Hons) in Computer Science}
		{National University of Singapore}
		{CAP: \textbf{4.45}/5.00, Honours (Distinction)}
\end{entrylist}

%----------------------------------------------------------------------------------------
%	ADDITIONAL INFORMATION
%----------------------------------------------------------------------------------------

\cvsect{Side projects}

\begin{entrylist}
	\entry
		{2020}
		{Reify\smallskip}{}
		{
			\smallskip
			\iconlight{Apple}{\href{https://apps.apple.com/us/app/reify-achieve-saving-goals/id1499847703}{iOS Appstore}}\hspace{1cm}
			\iconlight{Github}{\href{https://github.com/karrui/goals-flutter}{karrui/goals-flutter}} \\
			A goal tracking app built with Flutter for users to track their goals.
			Reify is an app that allows you to create virtual goals you can contribute to whenever you catch yourself having extra money. 
      Built to explore a new programming language, and to try out designing a "native" app all by myself.
			\\ \texttt{Flutter}\slashsep\texttt{Dart}\slashsep\texttt{Firebase}
		}
	\entry
		{2018}
		{CAP Calculator\smallskip}{}
		{
			\smallskip
			\iconlight{ExternalLinkSquare}{\href{https://cap.karrui.dev}{cap.karrui.dev}}\hspace{1cm}
			\iconlight{Github}{\href{https://github.com/karrui/cap-calculator}{karrui/cap-calculator}} \\
			A TypeScript React application for NUS undergraduates to calculate their current CAP (Cumulative Average Point), helping them
		to visualize what grades are needed to hit their goal, or what their current grade is.
		\\ \texttt{TypeScript}\slashsep\texttt{React}\slashsep\texttt{Firebase}\slashsep\texttt{Netlify}
		}
	\entry
		{2018}
		{Static Program Analyser For SIMPLELang\smallskip}{}
		{
			\smallskip
			\iconlight{Info}{Code privatised (due to school policy) but available on request} \\
			A C++11 desktop static program analyser application. Worked extensively on the tokenizer and the parser for parsing the given 
			input source code. Made sure code followed design principles such as Open-Closed Principle. Wrote unit, integration and system tests using equivalence partitioning.
		\\ \texttt{C++11}
		}
\end{entrylist}

% \begin{minipage}[t]{0.3\textwidth}
% 	\vspace{-\baselineskip} % Required for vertically aligning minipages

% 	\cvsect{Languages}
	
% 	\textbf{English} - native\\
% 	\textbf{German} - proficient\\
% 	\textbf{Polish} - rudimentary
% \end{minipage}
% \hfill
% \begin{minipage}[t]{0.3\textwidth}
% 	\vspace{-\baselineskip} % Required for vertically aligning minipages
	
% 	\cvsect{Hobbies}
	
% 	I love... \lorem
% \end{minipage}
% \hfill
% \begin{minipage}[t]{0.3\textwidth}
% 	\vspace{-\baselineskip} % Required for vertically aligning minipages
	
% 	\cvsect{Non profit}
	
% \end{minipage}

%----------------------------------------------------------------------------------------

\end{document}
