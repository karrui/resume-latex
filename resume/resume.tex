\documentclass{resume}
\usepackage[hidelinks]{hyperref}
\usepackage{microtype}
\newcommand{\mytilde}{\raise.17ex\hbox{$\scriptstyle\mathtt{\sim}$}}

\name{\underline{Lau} Kar Rui}
\contact{\href{tel:+6583221353}{+6583221353}}
\contact{\href{mailto:hi@karrui.dev}{hi@karrui.dev}}
\contact{\href{https://github.com/karrui}{github.com/karrui}}
\contact{\href{https://linkedin.com/in/karrui}{linkedin.com/in/karrui}}

\begin{document}
\makeheader

\begin{ResumeSection}{experience}
    \begin{ResumeSubsection}{org=\href{https://open.gov.sg}{Open Government Products},location={Singapore},position={Lead Software Engineer},duration=April 2024 - Present}
        \begin{itemize} 
            \item As one of the organisation's subject matter experts on frontend engineering, started a Frontend Readability Working Group to help review or give inputs on frontend engineering artifacts. 
            \item To uphold engineering quality (and continuing from previous efforts), I was in charge of setting organisation-wide testing expectations, which includes ensuring teams run their tests in continuous integration pipelines, monitoring their tests' flakeyness and coverage.
            \item Embedded myself into multiple teams across the organisation to raise (or maintain) engineering standards. This was done via mentoring (junior or otherwise) engineers, setting up project foundation and architecture, churning out code and code review.
        \end{itemize}
        \medskip
    \end{ResumeSubsection}
    \begin{ResumeSubsection}{org=,location=,position={Senior Software Engineer},duration=October 2021 - April 2024}
        \begin{itemize}
            \item Tech lead for the \href{https://activesg.gov.sg/}{new MyActiveSG+ application}, a webapp for Singapore's national movement to promote active living. Scoped, prioritised and contributed features, and managed stakeholders for 2 successful pilots and subsequent release to the public. The app easily handles \textbf{\mytilde150,000 facility/programmes/passes bookings per month}.
            \item Lead, scoped and contributed significantly to the migration of \href{https://github.com/opengovsg/FormSG}{FormSG}'s frontend framework from AngularJS to React 17 (latest version at the time) with \textbf{zero downtime}. FormSG is a form builder for the Singapore Government.
            \item Created and maintained our \href{https://design.open.gov.sg/}{internal design system} used across \textbf{70\% of organisation's current live projects}, released as an \href{https://www.npmjs.com/package/@opengovsg/design-system-react}{npm package}, achieving visual continuity across our products and reducing developer toil.
            \item Propagated and maintained our \href{https://start.open.gov.sg/}{organisation-wide starter kit template}, incorporating best practices and baseline-functionality to be used when bootstrapping a new product, allowing engineers to focus on product market fit instead of spending time on setting up the infrastructure. \textbf{70\% of all live (and experimental) projects in 2024 were bootstrapped with OGP Starter Kit}.
            \item Champion the use of Storybook and Chromatic as a visual regression testing and usage-guidelines tool across the organisation.
            \item Wrote lambda pipelines to automate parts of patient and case management tasks relating to Singapore’s COVID-19 Endemic response.
        \end{itemize}
        \medskip
    \end{ResumeSubsection}
    \begin{ResumeSubsection}{org=,location=,position={Software Engineer},duration=February 2020 - October 2021}
        \begin{itemize}
            \item Lead, scoped and completed full rehaul of FormSG’s backend API endpoints (with backwards compatibility) from JavaScript to Typescript, which includes renaming, simplifying, and refactoring endpoints to a more RESTful and meaningful state, now being fully type-safe.
            \item Built the entire clinician-facing frontend application for Singapore’s COVID-19 Appointment System to allow clinic staff to manage appointments and patients coming in for their vaccinations.
        \end{itemize}
        \bigskip
    \end{ResumeSubsection}
    \begin{ResumeSubsection}{org=Google,location={Singapore},position={Software Engineer Intern},duration=May 2019 - August 2019}
        \begin{itemize} 
            \item Delivered the bill-split feature in the (then) Android application, including but not limited to high level architecture, design decisions, implementation, and comprehensive testing
        \end{itemize}
        \bigskip
    \end{ResumeSubsection}
\end{ResumeSection}

\begin{ResumeSection}{education}
    \begin{ResumeSubsection}{org={National University of Singapore},position={Bachelor of Computing},duration={2016 -- 2019}}
            \\CAP: 4.45/5.00, Honours (Distinction)
    \end{ResumeSubsection}
\end{ResumeSection}

\begin{ResumeSection}{skills}
    \newcommand{\skill}[2]{\textbf{#1} - #2}
        \begin{itemize}
            \item \skill{Web Application Development}{Node.js, TypeScript, React.js, Next.js, CSS, Design Systems, SQL}
            \item \skill{Infrastructure}{Docker, AWS, Pulumi, Infrastructure as Code}
        \end{itemize}
\end{ResumeSection}

\begin{ResumeSection}{projects}
    \begin{itemize}
        \item Developed Reify (\href{https://apps.apple.com/us/app/reify-achieve-saving-goals/id1499847703}{available in the iOS Appstore}), an app that allows you to create virtual saving goals. \hfill\em{2020}
    \end{itemize}
\end{ResumeSection}

\end{document}